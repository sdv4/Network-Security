\documentclass[11pt]{article}
\usepackage{array, url, kantlipsum, listings, xcolor}
\usepackage[margin=1in]{geometry}

\lstdefinestyle{terminal}
{
    backgroundcolor=\color{white},
    basicstyle=\footnotesize\color{black}\ttfamily,
    frame=tb,
    numbers=left
}
\makeatletter
\long\def\@makecaption#1#2{%
	\vskip\abovecaptionskip
		\bfseries #1: #2\par
	\vskip\belowcaptionskip}%
\makeatother

\title{Assignment 2 \\CPSC 526 Fall 2017 \\ October 15, 2017}
\author{
\begin{tabular}{c c}
Mason Lieu & Shane Sims\tabularnewline
ID: 10110089 & ID: 00300601\tabularnewline
Tutorial 04 & Tutorial 04 \tabularnewline
\url{mlieu@ucalgary.ca} & \url{shane.sims@ucalgary.ca}
\end{tabular}}
\date{}

\begin{document}
\maketitle

\section*{How to run and connect to the backdoor}
\begin{lstlisting}[style=terminal, title={Running the backdoor}]
python3 backdoor_server.py <port number >
\end{lstlisting}
\begin{lstlisting}[style=terminal, title={Connecting to the backdoor}]
nc localhost <port number of the backdoor>
Enter Password: thepass
\end{lstlisting}

\section*{Handshake details}
The backdoor uses the TCP Internet protocol and this is implemented using the localhost address and a port provided by the user. A client can connect to the backdoor using a networking utility such as Netcat and providing the correct password. The backdoor will listen on the port for a client to connect to it.

\section*{Supported commands}
The supported commands are as follows
\begin{enumerate}
\item pwd
\item cd
\item ls
\item cp
\item mv
\item rm
\item cat
\item snap
\item diff
\item help
\item logout
\item off
\item who
\item ps
\end{enumerate}


\end{document}

